\documentclass[letter, 10pt]{article}
\usepackage[latin1]{inputenc}
\usepackage[spanish]{babel}
\usepackage{amsfonts}
\usepackage{amsmath}
\usepackage[dvips]{graphicx}
\usepackage{url}
\usepackage[top=3cm,bottom=3cm,left=3.5cm,right=3.5cm,footskip=1.5cm,headheight=1.5cm,headsep=.5cm,textheight=3cm]{geometry}


\begin{document}
\title{Inteligencia Artificial \\ \begin{Large}Estado del Arte: Vehicle Routing Problem \emph{(VRP)}\end{Large}}
\author{Elian Vallejos}
\date{\today}
\maketitle


%--------------------No borrar esta secci\'on--------------------------------%
\section*{Evaluaci\'on}

\begin{tabular}{ll}
Resumen (5\%): & \underline{\hspace{2cm}} \\
Introducci\'on (5\%):  & \underline{\hspace{2cm}} \\
Definici\'on del Problema (10\%):  & \underline{\hspace{2cm}} \\
Estado del Arte (35\%):  & \underline{\hspace{2cm}} \\
Modelo Matem\'atico (20\%): &  \underline{\hspace{2cm}}\\
Conclusiones (20\%): &  \underline{\hspace{2cm}}\\
Bibliograf\'ia (5\%): & \underline{\hspace{2cm}}\\
 &  \\
\textbf{Nota Final (100\%)}:   & \underline{\hspace{2cm}}
\end{tabular}
%---------------------------------------------------------------------------%
\vspace{2cm}


\begin{abstract}
Lo que se pretende presenta en el siguiente documento, es el problema \emph{Vehicle Routing Problem}.
\newline
Este problema es una variante del conocido \emph{Traveling Salesmen Problem (TSP)}, lo que aqu\'i se quiere realizar,
es satisfacer la demanda de los clientes, entregando se mercader\'ia, minimizando los costos asociados a las rutas de entrega. Para 
esto se cuenta con una flota de veh\'iculos, los que son todos homog\'eneos. Adem\'as de realizar la entrega, los veh\'iculos deben volver 
al punto de origen, o sea a su deposito. Lo m\'as importante en esta problem\'atica es poder satisfacer la necesidad de los clientes,
teniendo en cuenta los recursos con los cuales se cuenta, dado que no hay muchas variables en el problema, no es necesario ahondar m\'as
en esto.
A continuaci\'on se proceder\'a a definir el problema y buscar un m\'etodo que logre obtener una soluci\'on 
mas cercana al \'optimo. Para esto, se presentara un planteamiento formal, estado del arte del problema, posibles m\'etodos 
de soluci\'on y descripci\'on del modelo matem\'atico elegido.

\end{abstract}

\section{Introducci\'on}
Una explicaci\'on breve del contenido del informe. Es decir, detalla: Prop\'osito, Estructura del Documento, Descripci\'on (muy breve) del Problema y Motivaci\'on.

\section{Definici\'on del Problema}



\subsection{Traveling Salesman Problem (TSP)}

El problema del vendedor viajero es uno de los problemas m\'as estudiados, 
este problema responde a la pregunta, que dada una lista de ciudades 
junto con las distancias de esta, ?` Cual es la ruta mas corta posible 
para visitar cada ciudad exactamente una vez, y regresar a la ciudad de origen?. 
Este es un problema NP-duro ~\cite{TSP} adem\'as de ser netamente combinatorio. 
Porque mencionar este problema, dado que se puede decir que
es el problema madre para \emph{VRP}, pr\'acticamente es el mismo salvo que para el 
\emph{TSP} solo es un veh\'iculo, por ende se puede encontrar mucha m\'as literatura para 
la resoluci\'on de este problema, no se ha ahondara m\'as dado que abarca demasiado y solo 
es \'util su menci\'on y en que consiste.
\newline
Del \emph{TSP} se desprenden adem\'as variantes a este problema, que en este caso conviene 
mencionarlos igualmente dada su similitud con el \emph{VRP} ~\cite{TSP}

\begin{itemize}
 \item Traveling Salesman Problem with Backhauls (TSPB)
 \item Traveling Salesman Problem with Time Windows (TSPTW)
 \item Multiple Traveling Salesman Problem (MTSP)
\end{itemize}

En general todas variaciones anteriores se aplican de una manera al \emph{VRP}, dado que el \emph{TSP} no es lo que se quiere explicar, no hay necesidad de
detallar mas las variantes, de este.

\subsection{Capacitated Vehicle Routing Problem (CVRP)}
Esta variaci\'on al igual que el problema estudiado es una flota de veh\'iculos de reparto con una capacidad uniforme, los cuales 
tienen que atender la demanda conocida de cliente, en distintos puntos geogr\'aficos, satisfaciendo la restricci\'on de que las rutas descritas tienen
que ser a un m\'inimo costo tal cual lo har\'ia el \emph{VRP}, con la salvedad de que la capacidad de cada veh\'iculo es 
uniforme de una sola mercader\'ia~\cite{CVRP}.
\newline
El objetivo de esta variaci\'on del problema es que se quiere reducir al m\'inimo posible la flota de los veh\'iculos de reparto, adem\'as 
de la suma de los tiempos de viaje, considerando tambi\'en de que la capacidad de cada veh\'iculo no puede ser excedida por la ruta a tomar.
Su formulaci\'on es la misma que para el \emph{VRP} agregando la restricci\'on de que la demanda total de los clientes en una misma ruta
no debe sobrepasar la capacidad del veh\'iculo asignado a la misma.


\subsection{Distance Constrained Vehicle Routing Problem (DCVRP)}

Esta variante del problema radica en que a diferencia del \emph{Capacitated Vehicle Routing Problem (CVRP)} aqu\'i la restricci\'on de la capacidad
del veh\'iculo es cambiada por la distancia ~\cite{TSP} en que se tiene que hacer el recorrido, dado que existe un limite m\'aximo el cual 
tiene que tener la ruta. Sigue teniendo m\'ultiples veh\'iculos, punto de salida y final el mismo, etc.
Aqu\'i usualmente el costo se relaciona directamente con el viaje realizado o con el tiempo de este.


\subsection{Vehicle Routing Problem with Backhauls (VRPB)}

La variante \emph{VRPB} tambi\'en conocida como linehault-backhault problem, tiene la particularidad que un cliente, puede estar satisfecho o no
con el producto que se le entrega, dada esta premisa, este podr\'a devolver o no el producto al veh\'iculo repartidor ~\cite{Prosser93Hybrid}, 
dada esta problem\'atica
es que se incorporan como restricciones el contenedor del veh\'iculo como algo dimensional, ya que es necesario saber la capacidad 
de este adem\'as 
de si es necesario reordenar la carga, para as\'i poder llevar todo nuevamente al deposito ~\cite{Prosser93Hybrid}, sin que el veh\'iculo se quede sin capacidad en alg\'un 
momento.

\subsection{Vehicle Routing Problem with Time Windows (VRPTW)}

El \emph{VRPTW} se diferencia con los antecesores en que aqu\'i existe un periodo de tiempo en el cual el veh\'iculo debe iniciar el recorrido para
cada cliente, para as\'i llegar sin problemas en el tiempo esperado por el cliente ~\cite{journals/eor/AziGP07}.
Esta variante adem\'as toma en consideraci\'on, de que si el veh\'iculo llega antes del tiempo estimado, debe esperar un tiempo t para que empiece
nuevamente el servicio, de manera que los plazos se cumplan cabalmente.

\subsection{Vehicle Routing Problem with Pickup and Delivery (VRPPD)}

Aqu\'i lo que se modela es que cada cliente pide una cierta cantidad de productos, los cuales tienen que ser recogidos en el almac\'en, para luego ser
llevados por el veh\'iculo hasta el cliente. Cada veh\'iculo tiene una capacidad de carga, por lo que los productos deben caber dentro del veh\'iculo
para as\'i poder ser entregados al cliente, aqu\'i el problema es mucho mas realista en algunos puntos, al igual que sus antecesores se debe considerar 
la ruta, los clientes, y adem\'as poder minimizar los costos, ya sea ruta mas corta, cantidad de veh\'iculos empleados, para as\'i poder satisfacer 
la demanda de los clientes.
~\cite{EilamTzoreff2002193}.

\section{Estado del Arte}

\section{Modelo Matem\'atico}
Uno o m\'as modelos matem\'aticos para el problema, idealmente indicando el espacio de b\'usqueda para cada uno.
\subsection{Par\'ametros}
\subsection{Variables}
\subsection{Restricciones}
\subsection{Funci\'on objetivo}
\subsection{Espacio de B\'usqueda}

\section{Conclusiones}
Conclusiones RELEVANTES del estudio realizado.

\section{Bibliograf\'ia}
Indicando toda la informaci\'on necesaria de acuerdo al tipo de documento revisado. Todas las referencias deben ser 
citadas en el documento.
\bibliographystyle{plain}
\bibliography{Referencias}

\end{document} 