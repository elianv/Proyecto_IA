\documentclass[letter, 10pt]{article}
\usepackage[latin1]{inputenc}
\usepackage[spanish]{babel}
\usepackage{amsfonts}
\usepackage{amsmath}
\usepackage[dvips]{graphicx}
\usepackage{url}
\usepackage[top=3cm,bottom=3cm,left=3.5cm,right=3.5cm,footskip=1.5cm,headheight=1.5cm,headsep=.5cm,textheight=3cm]{geometry}


\begin{document}
\title{Inteligencia Artificial \\ \begin{Large}Estado del Arte: Vehicle Routing Problem\end{Large}}
\author{Elian  Vallejos}
\date{\today}
\maketitle


%--------------------No borrar esta secci\'on--------------------------------%
\section*{Evaluaci\'on}

\begin{tabular}{ll}
Resumen (5\%): & \underline{\hspace{2cm}} \\
Introducci\'on (5\%):  & \underline{\hspace{2cm}} \\
Definici\'on del Problema (10\%):  & \underline{\hspace{2cm}} \\
Estado del Arte (35\%):  & \underline{\hspace{2cm}} \\
Modelo Matem\'atico (20\%): &  \underline{\hspace{2cm}}\\
Conclusiones (20\%): &  \underline{\hspace{2cm}}\\
Bibliograf\'ia (5\%): & \underline{\hspace{2cm}}\\
 &  \\
\textbf{Nota Final (100\%)}:   & \underline{\hspace{2cm}}
\end{tabular}
%---------------------------------------------------------------------------%
\vspace{2cm}


\begin{abstract}
Resumen del informe en no m\'as de 10 l\'ineas.
\end{abstract}

\section{Introducci\'on}
Una explicaci\'on breve del contenido del informe. Es decir, detalla: Prop\'osito, Estructura del Documento, Descripci\'on (muy breve) del Problema y Motivaci\'on.

\section{Definici\'on del Problema}
Explicaci\'on del problema que se va a estudiar, en qu\'e consiste, cuales son sus variables, restricciones y objetivos de manera general.
Variantes m\'as conocidas que existen.

\section{Estado del Arte}
Lo m\'as importante que se ha hecho hasta ahora con relaci\'on al problema. Deber\'ia responder preguntas como las siguientes:
?`cuando surge el problema?, ?`qu\'e m\'etodos se han usado para resolverlo?, ?`cuales son los mejores algoritmos que se han creado hasta
la fecha?, ?`qu\'e representaciones han tenido los mejores resultados?, ?`cu\'al es la tendencia actual? tipos de movimientos,
heur\'isticas, m\'etodos completos, h\'ibridos, etc... Puede incluir gr\'aficos comparativos, o explicativos.\\
La informaci\'on que describen en este punto se basa en los estudios realizados con antelaci\'on respecto al tema.
Dichos estudios se citan de manera que quien lea su estudio pueda tambi\'en
 acceder a las referencias que usted revis\'o. Las citas se realizan mediante el comando \verb+\cite{ }+.
Por ejemplo, para hacer referencia al art\'iculo de algoritmos h\'ibridos para problemas de satisfacci\'on 
 de restricciones que ley\'o para el primer certamen~\cite{Prosser93Hybrid}.

\section{Modelo Matem\'atico}
Uno o m\'as modelos matem\'aticos para el problema, idealmente indicando el espacio de b\'usqueda para cada uno.

\section{Conclusiones}
Conclusiones RELEVANTES del estudio realizado.

\section{Bibliograf\'ia}
Indicando toda la informaci\'on necesaria de acuerdo al tipo de documento revisado. Todas las referencias deben ser 
citadas en el documento.
\bibliographystyle{plain}
\bibliography{Referencias}

\end{document} 